%        File: doc.tex
%     Created: Lun mar 29 09:00  2021 C
% Last Change: Lun mar 29 09:00  2021 C
%
\documentclass[a4paper]{article}

\usepackage[utf8]{inputenc}
\usepackage[english]{babel}
\usepackage[]{amsmath}

\newcommand{\rom}[1]{\mathrm{#1}} 
\newcommand{\bold}[1]{\mathbf{#1}} 
\newcommand{\dd}{\mathrm{d}} 

\begin{document}

\section{Slater-type orbitals}

A real Slater-type orbital (STO) is defined as follows: 

\begin{equation}
  \chi_{nlm}(\zeta, \bold{r}) = N_n(\zeta) R_n(r, \zeta) S_{lm}(\theta, \phi)
  \label{eq:real_sto}
\end{equation}

where

\begin{equation}
  N_n(\zeta) = \frac{(2\zeta)^{n+\frac{1}{2}}}{\sqrt{(2n)!}} 
  \label{eq:norm_sto}
\end{equation}

is the normalization factor,

\begin{equation}
  R_n(r,\zeta) = r^{n-1} e^{-\zeta r} 
  \label{eq:sto_radial_part}
\end{equation}

is the radial part and $S_{lm}(\theta, \phi)$ is a real spherial harmonic.

\section{Overlap integrals}

The one-center overlap integrals

\begin{equation}
  S_{nlm,n'l'm'}(\zeta, \zeta') = 
  \int \chi_{nlm}^{\star}(\zeta, \bold{r}) 
  \chi_{n'm'l'}(\zeta', \bold{r}) \dd V 
  \label{eq:1c_overlap_integral} 
\end{equation}

is the most straightforward molecular integrals and is readily evaluated in
spherical coordinates. The integration of the radial part yields

\begin{equation}
  \int_0^{\infty} R_n(\zeta,r) R_{n'}(\zeta',r) r^2 \dd r = 
  \frac{(n+n')!}{(\zeta+\zeta')^{n+n'+1}}
  \label{eq:1c_overlap_integral_radial_part_eval}
\end{equation}

and using the orthonormality condition of the spherical harmonics, the analytical
form of one-center overlap is

\begin{equation}
  S_{nlm,n'l'm'}(\zeta, \zeta') = 
  N_n(\zeta) N_{n'}(\zeta') 
  \frac{(n+n')!}{(\zeta+\zeta')^{n+n'+1}}
  \delta_{ll'}\delta_{mm'}
\label{eq:1c_overlap_integral_result}
\end{equation}

\end{document}


